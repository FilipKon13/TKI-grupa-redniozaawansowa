\documentclass[]{article}
\usepackage[polish]{babel}
\usepackage[T1]{fontenc}
\usepackage{geometry}
\usepackage{amsmath}

\newcommand{\Oh}[1]{\mathcal{O}{\left(#1\right)}}

\newgeometry{left=4cm, right=4cm}

\author{Filip Konieczny}
\date{4 lipca 2024}
\title{Omówienie zadania \texttt{Wycinek}}

\begin{document}
\maketitle
\section{Omówienie}

Mamy daną tablicę $(T_i)_{i=1}^n$ liczb całkowitych i chcemy znaleźć jej najdłuższy spójny fragment, którego elementy sumują się do $s$. Gdyby liczby podane na wejściu były nieujemne, wtedy zadziałałoby rozwiązanie oparte na gąsienicy: trzymamy dwa indeksy i sumę elementów między nimi. Jeśli suma elementów jest mniejsza badź równa $s$, to przesuwamy przedni indeks, jeśli za duża to tylni. Jeśli mamy sumę dokładnie $s$ to porównujemy aktualny przedział z najlepszym.

Ponieważ oba wskaźniki przesuniemy sumarycznie liniowo wiele razy, podejście to ma złożoność liniową. Niestety, w naszej sytuacji nie działa ono (istnieje wiele kontrprzykładów, zachęcam do znalezienia jakiegoś), bo powiększanie przedziału niekoniecznie zwiększa sumę elementów w nim.

Oznaczmy $pre_0 = 0, pre_i = T_1 + T_2 +\ldots + T_i$, tzn. tablicę sum prefiksowych. Zauważmy, że przedział $(i,j)$ ma sumę $s$ wtedy i tylko wtedy gdy $pre_j - pre_{i-1} = s$ i $j \geq i$. Będziemy kolejno rozważać wszystkie $j$-ty i znajdywać dla danego indeksu najlepsze (tzn. najmniejsze) pasujące $i$. Jak to zrobimy? 

Dla ustalonego $j$ chcemy znaleźć najmniejsze $i$ spełniające $pre_{i-1} = pre_j - s$. Jak to osiągnąć? Wszystkie napotkane do tej pory sumy prefiksowe trzymamy w mapie (hash-mapie) $M$, takiej, że $M[pre_{i-1}] = i$. Wtedy, najlepsze $i$ dla danego $j$ to $M[pre_j - s]$ (o ile istnieje).

Robimy tak dla każdego $j$ i aktualizujemy mapę odpowiednio (uwaga: nie należy nadpisywać pola w mapie, jeśli już ma jakąś wartość - szukamy najmniejszego indeksu o danej sumie). Rozwiązanie ma złożoność $\Oh{n\log{n}}$ lub oczekiwaną $\Oh{n}$ w zależności od tego czy użyjemy \texttt{std::map} czy \texttt{std::unordered\_map}.

\end{document}