\documentclass[]{article}
\usepackage[polish]{babel}
\usepackage[T1]{fontenc}
\usepackage{geometry}
\usepackage{amsmath}

\newgeometry{left=4cm, right=4cm}

\author{Filip Konieczny}
\date{4 lipca 2024}
\title{Omówienie zadania \texttt{bbi}}

\begin{document}
\maketitle
\section{Omówienie}

W zadaniu należy zamienić liczbę całkowitą nieujemną z wejścia na jej reprezentację w systemie dwójkowym (16 bitów).

Oczekiwanym rozwiązaniem jest użycie standardowego algorytmu na zamianę liczby na inną bazę: w przypadku bazy o podstawie dwa algorytm jest następujący:

\begin{itemize}
    \item Weź resztę z dzielenia przez 2 danej liczby; jest to kolejna cyfra wyniku;
    \item Podziel daną liczbę przez 2 (zaokrąglając w dół);
    \item Powtarzaj dopóki liczba jest większa od 0.
\end{itemize}

Powyższy ciąg operacja zwróci kolejne cyfry wyniku w kolejności od najmniej znaczącej (tj. cyfry jedności) do najbardziej znaczącej. Otrzymaną reprezentację należy uzupełnić zerami, aby otrzymać napis długości 16. Uwaga na przypadek, kiedy liczba jest równa 0.

Istnieje też krótsze (przynajmniej jeśli chodzi o kod) rozwiązanie, które korzysta z biblioteki standardowej, a konkretnie strucktury \texttt{std::bitset} (można ją znaleźć w nagłówku \texttt{<bitset>}). Typowo służy ona do utrzymywania masek bitowych w skompresowany sposób, by przyspieszyć wykonywanie na nich pewnych operacji.

W naszym przypadku pomocne będą dwie zaimplementowane przez \texttt{std::bitset} własności, mianowicie: przeciążony \texttt{operator}\texttt{<}\texttt{<} dla \texttt{std::ostream} (co znaczy tyle, że można \texttt{std::bitset} wypisać przy pomocy \texttt{std::cout}) oraz konstruktor przyjmujący liczbę całkowitą i tworzący \texttt{std::bitset} z jej reprezentacją binarną. Innymi słowy \texttt{std::bitset<16> bs(x);} tworzy bitset o wymaganej szerokości i z reprezentacją binarną liczby \texttt{x}. Wystarczy go więc po prostu wypisać.



\end{document}