\documentclass[]{article}
\usepackage[polish]{babel}
\usepackage[T1]{fontenc}
\usepackage{geometry}
\usepackage{amsmath}

\newcommand{\Oh}[1]{\mathcal{O}{\left(#1\right)}}

\newgeometry{left=4cm, right=4cm}

\author{Filip Konieczny}
\date{4 lipca 2024}
\title{Omówienie zadania \texttt{Książka telefoniczna}}

\begin{document}
\maketitle
\section{Omówienie}

Po przetłumaczeniu treści na język grafowy zadanie jest następujące: mamy podany graf oraz ciąg zapytań o pary wierzchołków. Dla każdego zapytania należy wypisać, czy oba wierzchołki znajdują się w jednej spójnej składowej.

Pokażemy dwa sposoby jak można rozwiązać to zadanie:

\subsection*{Przeglądanie grafu}

Możemy ponumerować składowe grafu, tak, żeby każdy wierzchołek wiedział, jaki jest numer składowej, w której się znajduje (efektywnie znaczy to, że wszystkie składowe zostały pokolorowane na różne kolory). Po tym odpowiadanie na zapytania jest już proste: dwa wierzchołki są w jednej składowej wtedy i tylko wtedy gdy mają ten sam kolor.

Kolorowanie/numerowanie składowych można zrealizować za pomocą dowolnego algorytmu przeglądania grafu, np. DFS lub BFS. Uzyskujemy algorytm działający w złożoności $\Oh{n + m + q}$, czyli liniowej.

\subsection*{Find-Union}

Struktura Find-Union także pozwala nam rozwiązać to zadanie. Dodanie krawędzi to operacja \texttt{Union} (o ile wierzchołki są w różnych zbiorach). Zapytanie czy są w jednej spójnej składowej to porównanie ich reprezentantów, których znajdujemy za pomocą operacji \texttt{Find}.

Uzyskane rozwiązanie działa w złożoności $\Oh{n+(m+q)T(n)}$, gdzie $T(n)$ to amortyzowany czas działania operacji \texttt{Find}, zależny od implementacji. Warto zauważyć, że to rozwiązanie jest poprawne, kiedy krawędzie oraz zapytania są wymieszane, tzn. musimy odpowiadać na zapytania na grafie, któremu może przybywać krawędzi. 


\end{document}