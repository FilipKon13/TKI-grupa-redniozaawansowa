\documentclass[]{article}
\usepackage[polish]{babel}
\usepackage[T1]{fontenc}
\usepackage{geometry}
\usepackage{amsmath}

\newcommand{\Oh}[1]{\mathcal{O}{\left(#1\right)}}

\newgeometry{left=4cm, right=4cm}

\author{Filip Konieczny}
\date{4 lipca 2024}
\title{Omówienie zadania \texttt{Liczby drugie}}

\begin{document}
\maketitle
\section{Omówienie}

Liczba jest \textit{druga}, gdy jej zapis w systemie dziesiętnym jest konkatenacją zapisów w systemie dziesiętnym dwóch liczb pierwszych. Na wejściu mamy liczbę i mamy wypisać czy jest ona \textit{druga}.

Liczby na wejściu są rzędu maksymalnie $10^{13}$. Jeśli zastanowić się nad tym chwilę, to oznacza to, że liczba podana na wejściu ma zapis dziesiętny długości co najwyżej 14. Jedna z kluczowych obserwacji w tym zadaniu, to fakt, że nie mamy zbyt dużo podziałów do przetestowania: w najgorszym wypadku zapis dziesiętny liczby na wejściu można podzielić na 13 sposobów!

Dla każdego podziału musimy sprawdzić czy jest poprawny, tj. czy obie uzyskane w ten sposób zapisy są poprawne (tj. nie mają zer wiodących) oraz czy reprezentują liczby pierwsze. Ponieważ liczby na wejściu są maksymalnie rzędu $10^{13}$, to spokojnie możemy użyć algorytmu działającego w czasie $\Oh{\sqrt{n}}$ sprawdzający czy liczba jest pierwsza.

Otrzymany w ten sposób algorytm sprawdza czy liczba $n$ zapisana na wejściu jest \textit{druga} w czasie $\Oh{\sqrt{n}\log{n}}$, co jest wystarczające w tym zadaniu.

Pozostaje kilka kwestii implementacyjnych. Trzeba umieć generować podziały liczby z wejścia i testować, czy są to reprezentacje liczb pierwszych. Aby zaadresować pierwszy z nich, można użyć metody \texttt{substr} na obiekcie \texttt{std::string}, która zwraca podnapis danego napisu. Z kolei, aby przetestować czy reprezentacja jest liczbą pierwszą, to 

\begin{itemize}
    \item Należy sprawdzić czy nie zaczyna się zerem,
    \item Należy sprawdzić czy jest liczbą pierwszą -- do tego można użyć wspomnianego algorytmu pierwiastkowego oraz użyć \texttt{std::stoll} aby zamienić napis na odpowiadający \texttt{long long} (wynik może nie zmieścić się w typie \texttt{int}).
\end{itemize}


\end{document}